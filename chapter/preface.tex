\chapter{前言}

\section{为什么要写这款教程}
因为自家家长对计算机的使用产生的困难,而且是很大方面的完全不了解。并且观察到市面上没有适合中老年人阅读的,针对如何学会使用现阶段电子设备的文章,这类文章对于这类人群来讲十分必要,在不了解功能的情况很容易误点导致错误情况的发生,所以我们决定利用自己所学的知识再加上一些参考资料,尽可能全面细致的写出针对中老年经常使用的产品(台式机,手机)的使用教程,当然,这款教程也适合于不太熟悉某些功能的适龄人群,帮助其更加快速的了解一些功能。

\section{教程特点和主要入手方向}
那么本教程从台式机的硬件来入手,再到电脑的功能键、组合键、快捷键,具体按哪里有什么样的效果。我们会用图文方式来进行细致的描述,考虑到中老年的视力情况,我们在具体情况下会录制简单的操作视频来保证教程的简单性,可读性,可操作性。接着会用一块章节具体介绍电脑内磁盘的具体作用,文件和一些常用的后缀名主要代表什么意思。另外,很多上班族出于工作的需要经常需要用U盘拷贝文件,转存图片,或者车行车载音乐的下载和成功应用,这些在没有具体了解过科技设备的中老年上班族看来十分困难的事情,只是因为没有人正确的引导具体如何做,本款教程恰可以提供这种指导。


对于现在盛行的QQ,微信等社交软件,很多中老年人仅止步于社交聊天,但是像截图,下载内容等却不怎么了解,本教程也会提供细致的指导。

至于手机的使用,很多中老年人也没有将其用到十分之一的作用,止步于“我这么大了拿着手机我能捣鼓什么东西呢?”可事实上,很多东西的确需要了解,我们会针对我们遇见的事例,提供图文支持,帮助中老年人快速入手。

\section{教程过后会遇到的困难}
中老年人学习能力和探索能力明显下降,面对大片文字反而会更加厌倦觉得没有必要,因此我们只能做到讲解的更加生动有趣直切要点,但是却不能让他们主动学习了解,这就希望青壮年能够普及科技意识,让他们使用过程能够更加智能化,理性化。


\section{规划和改进}
开源项目,希望更多的人可以出于公益加入我们,为了普及到中老年人,适应当下科技的改革和发展。更好地适应当前的电子设备,达到游刃有余的程度。同时这么做也让当下的青壮年减轻一定的负担,让目前正在接受教育的少年儿童能够跟从教程所指向更快速的了解科技产品的发展。

这是长期且需要坚持的教程,如果有合适的内容想要补充,可以将所想到的内容(最好结合图片)发至 489765924@qq.com 邮箱内,欢迎有想法的人参与编写!